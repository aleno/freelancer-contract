% !TEX TS-program = xelatex
% !TEX encoding = UTF-8
% vim: tw=79

\documentclass[10pt,a4paper,parskip]{scrartcl}

\usepackage{geometry}
\geometry{tmargin=1in,bmargin=1in,lmargin=1.4in,rmargin=1.4in}

\usepackage[swedish]{babel}

% character encoding
\usepackage{fontspec}

\usepackage{xunicode}
\usepackage{xltxtra}

% spacing
\frenchspacing
\setlength{\parskip}{12pt}
\setlength{\parindent}{0pt}

\begin{document}

\begin{center}
{\Large Konsult avtal}
\end{center}

[ LOGO ]

[ ADDRESS ]

[ TELEFON ]

[ KONTAKT E-POST ]

[ DATUM ]

\section{Sammanfattning:}

Vi gör alltid vårt bästa för att uppfylla dina krav och dina mål,
men ibland är det bäst att ha några enkla saker nedskrivna så att vi båda vet
vad som är vad, vem som ska göra vad och vad som händer om saker går fel. I
detta avtal kommer du inte hitta komplicerade juridiska termer eller stora
stycken med oläslig text. Vi har ingen önskan att lura dig att underteckna
något som du kanske senare ångrar. Vi vill ha vad som är bäst för båda parter,
nu och i framtiden.

\section{Kort sagt:}

Du ( [KUNDENS BOLAG] ) anställer oss ( [MITT BOLAG] ) till [PROJEKT NAMN] vid
timkostnad på [TIMLÖN] per timme. Visst är det lite mer komplicerat, men det
kommer vi till.

\section{Vad båda parter är överens om att göra?}

Som kund har du möjlighet och förmåga att ingå detta avtal på uppdrag av ditt
företag eller organisation. Du samtycker till att förse oss med allt vi
behöver för att slutföra projektet, inklusive text, bilder och annan
information om och när vi behöver det, och i det format som vi ber om. Du
samtycker till att granska vårt arbete, ge feedback och godkännande i tid.
Deadlines fungerar på två sätt och du kommer även att vara bunden till alla
datum som vi satt upp. Du godkänner också att hålla dig till den betalningsplan
som anges i slutet av detta kontrakt.

Vi har erfarenhet och förmåga att utföra de tjänster du behöver från oss och vi
kommer att genomföra dessa på ett professionellt och snabbt sätt. Längs vägen
kommer vi att sträva efter att uppfylla alla de tidsfrister som fastställts,
men vi kan inte vara ansvarig för ett missat lanseringsdatum eller en deadline,
om du har varit sen med att leverera material, inte har godkänt eller
undertecknat vårt arbete i tid i något skede. Utöver detta kommer vi också
upprätthålla sekretessen för all information som du ger oss.

\section{Låt oss gå in på lite praktiska detaljer}

\subsection{Vad vi behöver från dig}

Nedan finns en lista med resurser som vi behöver få från dig innan arbetet
kan påbörjas. Varje projekt är lite annorlunda, så det kan finnas
saker som vi har missat i listan, men vi kommer att meddela dig detta så fort
vi kan, om vi har glömt något vi behöver.

[ LISTA BEHOV ] ( inkluderar referenser , tillgång till nödvändiga tjänster
eller tillgångar , etc )

\subsection{Webbläsarkompabilitet}

Vi har enats om att testa vår kod i IE10+ lika så de senaste versionerna av
Opera, Safari och Chrome. Dessutom testar vi på mobila Safari och mobila Chrome
(iOS-enheter, Android-enheter). Om du behöver andra än de som anges testade
webbläsarna, vänligen förse oss med en lista på webbläsare och enheter. Att
testa ytterligare webbläsare kräver mer testtid samt eventuellt behov att koda
för den specifika webbläsare. Vi kodar allt med progressiv förbättring i
åtanke. Det innebär att moderna webbläsare får ha mindre skillnader mot äldre
webbläsare (till exempel text- och boxskuggor, rundade hörn, etc.) baserat på
vad en webbläsare kan göra. Alla ändringar kommer inte att påverka
användarupplevelsen.

\subsection{Testning}

Vi skriver tester innan vi skriver vår kod, som gör det möjligt för oss att se
till att all nödvändig funktionalitet inte bara finns där, utan även att den
fungerar oavsett hur mycket ändringar vi gör i programvaran. Om dina anställda
eller andra entreprenörer kommer att arbeta med projektet med oss,
kräver vi att de också skriva tester för all kod som de producerar för att se
till att vi alla är på samma blad och inte trampa varandra på fötterna.

\subsection{Versionshantering}

Git tillåter oss att hålla reda på alla förändringar som sker i projektet. På
detta sätt, om en bugg introduceras kan vi snabbt spåra vart den introducerades
och återställa ändringarna till det senast felfria tillståndet medan vi fixar
saker! Vi kräver att kod ska vara versionshanterad så att vi kan samarbeta med
dina anställda eller andra entreprenörer i projektet, utan rädsla för att vi
ångrar eller skriver över varandras ändringar.

\subsection{Tidsplan}

Projektet är planerat att börja [START DATUM], och kommer att fortgå till och
med [SLUTDATUM / Alt. PROJEKTTID].

Nedan listas de milstolpar som finns i tidsplanen:

[ MILSTOLPE, dagar som krävs ] Detta kontrakt kommer att betraktas
slutfört när allt arbete som utförts har betalats in i sin helhet.

\subsection{Ytterligare utveckling}
Eventuell ytterligare utveckling som behövs utöver det som vi har kommit
överens om att här måste åtföljas av ytterligare ett avtal som vi med glädje
kommer att tillhandahållas på begäran. Vi förbehåller oss rätten att vägra
ytterligare arbete som det inte uttryckligen beskrivs i detta dokument.

\section{Kommunikation}
All kommunikation kommer att göras regelbundet under affärstimmar
(måndag - fredag, 9-15 Svensk tid, med undantag för helgdagar) och vi kommer
besvara all korrespondens inom en arbetsdag om vi inte har notifierat dig att
vi kommer vara otillgängliga. Vi förstår att kriser uppstår och i så fall
kommer vår timkostnad att vara [TIMKOSTNAD FÖR KRISER]. Vi uppskattar
kommunikation i ett textbaserat medium men förstår att inte alla diskussioner
inte passar i skriven form. Vi kommer gladligen prata med dig via telefon,
Skype/Google Hangout eller i person om du är i [MIN STAD] området. Alla beslut
som tas kommer att skrivas ner och kräva ditt skriftliga godkännande.

Eftersom vi värdesätter både din och vår egen tid, ber vi att alla möten
planeras minst två arbetsdagar i förväg och följs av en dagordning så att vi
kan vara helt förberedda. Om vi inte har möjlighet att delta i ett möte på din
föreslagna tid, återkommer vi med två andra förslag som passar oss. Möten
påbörjas och avslutat på angiven tid eller innan den tilldelade tiden och
kommer att debiteras oavsett om du deltar eller inte.

\subsection{Projektledning}

Så länge du inte har ett föredraget projekthanteringsverktyg, kommer vi att
förse dig med en inloggning till vårt, där kommer du kunna se status för
projekt ner till den specifika uppgiften, så vilket vi kommer se till att
informationen är aktuell. Vi kommer även att förlita oss på att du deltar genom
att svara på våra frågor, lämna ditt godkännande, begära ändringar, och
verifiera “fullständighet” genom verktygen. Vi ber om en handläggningstid på en
arbetsdag för svar så att vi kan se till vi avklarar milstolparna som vi
beslutat om. För att hålla projektet organiserat och hanteringen enklare,
föredrar vi att du inte lämna in någon av de uppgifter som angetts ovan via
vanlig e-post.

\section{Rättsliga saker}
Även fast vi testar utförligt kan vi inte garantera att alla funktioner i någon
programvara alltid  kommer vara felfri så vi inte kan hållas ansvarig inför dig
eller någon tredje part för skador. Inklusive utebliven vinst, förlorat kapital
eller annan följdskada som uppstår vid drift eller oförmåga att använda
programvaran och eller andra webbplatser. Även om du har meddelat oss om
möjligheten för sådana skador.

Om någon föreskrift i detta avtal anses vara olagligt, ogiltigt eller av någon
anledning inte kan tillämpas, då ska den föreskriften bortses från detta avtal
och inte påverka giltigheten och genomförande av övriga föreskrifter i detta
avtal.


\section{Upphovsrätt}
Du garanterar oss att all text, grafik, fotografier, mönster, varumärken eller
andra illustrationer som du förser oss med för att inkludera i programvaran är
antingen ägd av dig eller att du har tillstånd att använda dem.

När vi mottagit din slutgiltiga betalning, är upphovsrätten automatiskt
tilldelad enligt följande:

Du äger grafik och andra visuella element som vi skapar till dig för detta
projekt. Vi kommer att ge dig en kopia av alla filer och du borde lagra dom
säkert eftersom vi inte är skyldig att behålla eller förse dig med
ursprungsformat som användes vid skapandet.

Du äger textinnehåll, fotografier och annan uppgifter som du har
tillhandahållit, förutom när någon annan äger den.

Vi äger källkoden vi har producerat och vi behåller rätten att licensiera det
hur vi vill, eller att bidra tillbaka till det open source-projekt som
programvaran baseras på.

Vi älskar att visa upp vårt arbete och att dela med oss till andra, så vi
förbehåller oss rätten att visa och länka till ditt färdiga projekt som en av
vår portfölj och att skriva om projektet på webbplatser, i tidningsartiklar, i
böcker och prata fritt om projektet på konferenser. Självklart lovar vi att
hålla den här informationen för oss själva till projektet lanseras eller 6
månader från leveransdagen, beroende på vilket som inträffar först.

\section{Betalningar}

Vi vet at du förstår hur viktigt det är för ett litet företag att du raskt
betalar de fakturor som vi skickar till dig. Eftersom vi även är säker på att
du vill att vi ska förbli vänner, så samtycker du att hålla dig till följande
betalningsplan.

\subsection{Betalningsplan}
Vi fakturerar varannan vecka på fredag och alla fakturor betalas vid
mottagandet.

Innan arbete påbörjas ska en förskottsbetalning på [FÖRSKOTTSBELOPP] betalas,
detta belopp kommer att dras från den sista fakturan. Om en betalning inte har
mottagits inom 7 dagar kommer arbetet att stoppas tills dess att betalningen
har mottagits. Om betalningen är försenad två gånger under projektets gång, så
förbehåller vi oss rätten att begära förskottsbetalning av timmar eller att
avsluta projektet, beroende på vad som anses mest passande för situationen.

Om någon av parterna väljer att avsluta projektet, måste det ske skriftligt.
Kom ihåg att förskottsbetalningen kommer att förverkas och allt betalt arbete
kommer att lämnas över till dig.

\section{Men vart är det hemskt finstilta?}
Du kan inte överlåta detta avtal till någon annan utan vårt tillstånd. Detta
avtal är giltigt och behöver inte förnyas. Om för någon anledning en del av
detta avtal blir ogiltigt eller inte kan göras gällande, så är de kvarvarande
delarna giltiga.

Även fast språket är enkelt, avsikterna är allvarliga och detta avtal är ett
juridiskt dokument under svensk lag.


\vspace{1cm} 

\noindent De undertecknade godkänner om de föreskrifter i detta avtal på
uppdrag av hans eller dess företag eller organisation.\\\\

\noindent \begin{tabular}{l l l}
På uppdrag av kunden: & \rule{6cm}{.2pt}       & Datum: \rule{2.4cm}{.2pt}\\
                      & [KUNDENS REPRESENTANT] & \\\\\\
Konsulten:            & \rule{6cm}{.2pt}       & Datum: \rule{2.4cm}{.2pt}\\
                      & [MITT NAMN]            & \\
\end{tabular}

\end{document}
